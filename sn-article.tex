%Version 3.1 December 2024
% See section 11 of the User Manual for version history
%
%%%%%%%%%%%%%%%%%%%%%%%%%%%%%%%%%%%%%%%%%%%%%%%%%%%%%%%%%%%%%%%%%%%%%%
%%                                                                 %%
%% Please do not use \input{...} to include other tex files.       %%
%% Submit your LaTeX manuscript as one .tex document.              %%
%%                                                                 %%
%% All additional figures and files should be attached             %%
%% separately and not embedded in the \TeX\ document itself.       %%
%%                                                                 %%
%%%%%%%%%%%%%%%%%%%%%%%%%%%%%%%%%%%%%%%%%%%%%%%%%%%%%%%%%%%%%%%%%%%%%

%%\documentclass[referee,sn-basic]{sn-jnl}% referee option is meant for double line spacing

%%=======================================================%%
%% to print line numbers in the margin use lineno option %%
%%=======================================================%%

%%\documentclass[lineno,pdflatex,sn-basic]{sn-jnl}% Basic Springer Nature Reference Style/Chemistry Reference Style

%%=========================================================================================%%
%% the documentclass is set to pdflatex as default. You can delete it if not appropriate.  %%
%%=========================================================================================%%

%%\documentclass[sn-basic]{sn-jnl}% Basic Springer Nature Reference Style/Chemistry Reference Style

%%Note: the following reference styles support Namedate and Numbered referencing. By default the style follows the most common style. To switch between the options you can add or remove “Numbered” in the optional parenthesis. 
%%The option is available for: sn-basic.bst, sn-chicago.bst%  
 
%%\documentclass[pdflatex,sn-nature]{sn-jnl}% Style for submissions to Nature Portfolio journals
%%\documentclass[pdflatex,sn-basic]{sn-jnl}% Basic Springer Nature Reference Style/Chemistry Reference Style
\documentclass[pdflatex,sn-mathphys-num]{sn-jnl}% Math and Physical Sciences Numbered Reference Style
%%\documentclass[pdflatex,sn-mathphys-ay]{sn-jnl}% Math and Physical Sciences Author Year Reference Style
%%\documentclass[pdflatex,sn-aps]{sn-jnl}% American Physical Society (APS) Reference Style
%%\documentclass[pdflatex,sn-vancouver-num]{sn-jnl}% Vancouver Numbered Reference Style
%%\documentclass[pdflatex,sn-vancouver-ay]{sn-jnl}% Vancouver Author Year Reference Style
%%\documentclass[pdflatex,sn-apa]{sn-jnl}% APA Reference Style
%%\documentclass[pdflatex,sn-chicago]{sn-jnl}% Chicago-based Humanities Reference Style

%%%% Standard Packages
%%<additional latex packages if required can be included here>

\usepackage{graphicx}%
\usepackage{multirow}%
\usepackage{amsmath,amssymb,amsfonts}%
\usepackage{amsthm}%
\usepackage{mathrsfs}%
\usepackage[title]{appendix}%
\usepackage{xcolor}%
\usepackage{textcomp}%
\usepackage{manyfoot}%
\usepackage{booktabs}%
\usepackage{algorithm}%
\usepackage{algorithmicx}%
\usepackage{algpseudocode}%
\usepackage{listings}%
%%%%

%%%%%=============================================================================%%%%
%%%%  Remarks: This template is provided to aid authors with the preparation
%%%%  of original research articles intended for submission to journals published 
%%%%  by Springer Nature. The guidance has been prepared in partnership with 
%%%%  production teams to conform to Springer Nature technical requirements. 
%%%%  Editorial and presentation requirements differ among journal portfolios and 
%%%%  research disciplines. You may find sections in this template are irrelevant 
%%%%  to your work and are empowered to omit any such section if allowed by the 
%%%%  journal you intend to submit to. The submission guidelines and policies 
%%%%  of the journal take precedence. A detailed User Manual is available in the 
%%%%  template package for technical guidance.
%%%%%=============================================================================%%%%

%% as per the requirement new theorem styles can be included as shown below
\theoremstyle{thmstyleone}%
\newtheorem{theorem}{Theorem}%  meant for continuous numbers
%%\newtheorem{theorem}{Theorem}[section]% meant for sectionwise numbers
%% optional argument [theorem] produces theorem numbering sequence instead of independent numbers for Proposition
\newtheorem{proposition}[theorem]{Proposition}% 
%%\newtheorem{proposition}{Proposition}% to get separate numbers for theorem and proposition etc.

\theoremstyle{thmstyletwo}%
\newtheorem{example}{Example}%
\newtheorem{remark}{Remark}%

\theoremstyle{thmstylethree}%
\newtheorem{definition}{Definition}%

\raggedbottom
%%\unnumbered% uncomment this for unnumbered level heads

\begin{document}

\title[Article Title]{Article Title}

%%=============================================================%%
%% GivenName	-> \fnm{Joergen W.}
%% Particle	-> \spfx{van der} -> surname prefix
%% FamilyName	-> \sur{Ploeg}
%% Suffix	-> \sfx{IV}
%% \author*[1,2]{\fnm{Joergen W.} \spfx{van der} \sur{Ploeg} 
%%  \sfx{IV}}\email{iauthor@gmail.com}
%%=============================================================%%

\author*[1,2]{\fnm{First} \sur{Author}}\email{iauthor@gmail.com}

\author[2,3]{\fnm{Second} \sur{Author}}\email{iiauthor@gmail.com}
\equalcont{These authors contributed equally to this work.}

\author[1,2]{\fnm{Third} \sur{Author}}\email{iiiauthor@gmail.com}
\equalcont{These authors contributed equally to this work.}

\affil*[1]{\orgdiv{Department}, \orgname{Organization}, \orgaddress{\street{Street}, \city{City}, \postcode{100190}, \state{State}, \country{Country}}}

\affil[2]{\orgdiv{Department}, \orgname{Organization}, \orgaddress{\street{Street}, \city{City}, \postcode{10587}, \state{State}, \country{Country}}}

\affil[3]{\orgdiv{Department}, \orgname{Organization}, \orgaddress{\street{Street}, \city{City}, \postcode{610101}, \state{State}, \country{Country}}}

%%==================================%%
%% Sample for unstructured abstract %%
%%==================================%%

\abstract{Spatial Gaussian fields (SGFs) are essential tools in spatial and spatio-temporal modeling, with the Matérn model frequently applied due to its flexibility and computational advantages. However, standard SGFs assume stationarity and isotropy, conditions that become unrealistic in environments with physical barriers. The Barrier model addresses this limitation but assumes that barriers are fully impermeable, a significant restriction in real-world applications. To resolve this, we propose the Transparent Barrier model, an innovative approach incorporating barriers with varying permeability, thus providing greater modeling flexibility and realism. We illustrate the utility of this model with marine megafauna distribution, specifically Dugongs (Dugong dugon), in the Red Sea, demonstrating its potential to accurately reflect complex spatial structures involving partially permeable barriers while maintaining computational efficiency.}

%%================================%%
%% Sample for structured abstract %%
%%================================%%

% \abstract{\textbf{Purpose:} The abstract serves both as a general introduction to the topic and as a brief, non-technical summary of the main results and their implications. The abstract must not include subheadings (unless expressly permitted in the journal's Instructions to Authors), equations or citations. As a guide the abstract should not exceed 200 words. Most journals do not set a hard limit however authors are advised to check the author instructions for the journal they are submitting to.
% 
% \textbf{Methods:} The abstract serves both as a general introduction to the topic and as a brief, non-technical summary of the main results and their implications. The abstract must not include subheadings (unless expressly permitted in the journal's Instructions to Authors), equations or citations. As a guide the abstract should not exceed 200 words. Most journals do not set a hard limit however authors are advised to check the author instructions for the journal they are submitting to.
% 
% \textbf{Results:} The abstract serves both as a general introduction to the topic and as a brief, non-technical summary of the main results and their implications. The abstract must not include subheadings (unless expressly permitted in the journal's Instructions to Authors), equations or citations. As a guide the abstract should not exceed 200 words. Most journals do not set a hard limit however authors are advised to check the author instructions for the journal they are submitting to.
% 
% \textbf{Conclusion:} The abstract serves both as a general introduction to the topic and as a brief, non-technical summary of the main results and their implications. The abstract must not include subheadings (unless expressly permitted in the journal's Instructions to Authors), equations or citations. As a guide the abstract should not exceed 200 words. Most journals do not set a hard limit however authors are advised to check the author instructions for the journal they are submitting to.}

\keywords{keyword1, Keyword2, Keyword3, Keyword4}

%%\pacs[JEL Classification]{D8, H51}

%%\pacs[MSC Classification]{35A01, 65L10, 65L12, 65L20, 65L70}

\maketitle

\section{Introduction}\label{sec1}

% \subsection{Background}

Spatial Gaussian fields (SGFs) are widely used in modeling spatial and spatio-temporal phenomena, particularly in applications where residual spatial structures arise due to unmeasured covariates, spatial aggregation, or spatial noise. Among these, the Matérn model is a prominent choice due to its flexibility and diverse applications. Advances such as the integrated nested Laplace approximation (INLA) (Rue et al., 2009) and stochastic partial differential equation (SPDE) approach (Lindgren et al., 2011) also support the use of the Matérn model by enabling efficient Bayesian inference and computational feasibility.

One weakness of SGFs like the Matérn model is their reliance on assumptions of stationarity and isotropy, implying spatial autocorrelation modeled through the random effect remains unchanged when the map is rotated. However, these assumptions become unrealistic in the presence of physical barriers, boundaries, or irregular features, such as coastlines or islands, where spatial dependency should not be based solely on the shortest Euclidean distance. Such cases require models that account for the effects of physical barriers.

The Barrier model was proposed to address this challenge by extending the Matérn framework to non-stationary settings. Instead of relying on the shortest Euclidean distance to determine spatial dependence, the Barrier model accounts for all potential connections between two locations. When physical barriers are present, the paths that cross over the barrier are removed, weakening the overall dependency between the two locations.

While the original Barrier model is limited to physically impermeable barriers, many real-world scenarios involve barriers with varying permeability. To address this, the Transparent Barrier model has been developed, introducing a framework to account for barriers with partial permeability. For instance, islands may act as impermeable barriers for marine species, whereas sand patches with tidal water coverage may allow partial movement. The Transparent Barrier model extends the original by incorporating transparency as a parameter controlling permeability, enabling it to handle both fully impermeable and partially permeable barriers within the same model. The latter class of barriers will be called transparent barriers.

This research introduces the Transparent Barrier model and demonstrates its application in real-world spatial settings. It retains the computational efficiency of stationary models while addressing the complexity of spatial structures influenced by barriers of varying nature, making it a practical and versatile tool for spatial modeling in marine science and beyond.

\section{Motivating Example}

Marine megafauna, such as whales, sharks, and large rays, play crucial roles in marine ecosystems, often regarded as keystone species due to their significant impact on the abundance and distribution of other organisms. Dugongs, for instance, help maintain seagrass ecosystems by grazing on seagrass beds, preventing overgrowth and promoting habitat complexity that supports diverse marine life. Therefore, conservation efforts focused on marine megafauna are essential for preserving marine ecosystem resilience.

Species distribution models (SDMs) provide valuable insights into species distributions, habitat preferences, and ecological interactions. Common SDMs predict species occurrences using environmental variables like temperature and ocean currents, employing algorithms such as MaxEnt, Random Forest, and Generalized Additive Models (GAMs). However, their effectiveness relies heavily on comprehensive environmental data. When such data are lacking, incorporating spatial random effects to account for unexplained spatial dependencies becomes essential, capturing variations in species distribution beyond environmental factors.

This study models Dugong distribution along the northern coast of the Saudi Arabian Red Sea, an area characterized by numerous islands and varying barrier permeability. Due to imprecise maps grouping islands rather than detailing their locations, available bathymetry data were used to construct accurate representations of islands rather than as direct covariates. This approach addresses limitations in traditional methods reliant on extensive environmental data, which were not applicable due to data constraints.

The Dugong data used are incidental sightings, collected opportunistically from activities such as tourism and citizen science initiatives. A Poisson process framework is employed as a suitable modeling approach for these randomly occurring events. Bathymetry data were provided by the Red Sea Global (RSG) project. By developing a model that accounts for spatial random effects and varying permeability of physical barriers, this research aims to support the conservation of Dugongs and other marine megafauna in the Red Sea region, serving as a guide for similar studies requiring advanced spatial modeling approaches.


\section{Methodology}

\subsection{Background}

In spatial statistics, the Matérn covariance model is frequently employed for representing Spatial Gaussian Fields (SGFs) \citep{Whittle1954, Stein1999, Diggle2010}. Its popularity is largely due to the INLA-SPDE framework \citep{Rue2009, Lindgren2011}, which mitigates the computational challenges associated with large covariance matrices by utilizing sparse precision matrices. This has led to widespread adoption of the Matérn covariance not only in Gaussian likelihood settings but also in complex scenarios such as marked point pattern models \citep{Illian2012}.

The Matérn covariance function for a spatial random process $\mu(s)$, defined as a function of the distance $\|s_i - s_j\|$ between locations $s_i$ and $s_j$, is given by:

\begin{equation}
\mathrm{Cov}(\mu(s_i), \mu(s_j)) = \sigma_{\mu}^{2}\frac{2^{1-\nu}}{\Gamma(\nu)}\left(\kappa\|s_i - s_j\|\right)^{\nu}K_{\nu}\left(\kappa\|s_i - s_j\|\right),
\end{equation}

where $\sigma_{\mu}^{2}$ is the marginal variance, $K_{\nu}$ is the modified Bessel function of the second kind, and $\nu > 0$, $\kappa > 0$ control smoothness and scale, respectively.

By setting $\nu = 1$ and adopting the empirical definition $\kappa = \sqrt{8\nu}/r$, we obtain:

\begin{equation}
\mathrm{Cov}(\mu(s_i), \mu(s_j)) = \sigma_{\mu}^{2}\left(\frac{\sqrt{8}}{r}\|s_i - s_j\|\right) K_{1}\left(\frac{\sqrt{8}}{r}\|s_i - s_j\|\right),
\end{equation}

where the range parameter $r = \sqrt{8\nu}/\kappa$ indicates the distance at which the spatial correlation between two points approximates $0.1$.

Since the Matérn covariance depends on the Euclidean distance, the SGF is assumed stationary and isotropic, meaning the spatial process is invariant to translation and rotation.

The Barrier model introduced by \citet{Bakka2019} extends the INLA-SPDE framework to non-stationary scenarios by assigning distinct Matérn fields to barrier and non-barrier areas. In the Barrier model, separate Matérn fields are specified for normal and barrier regions. Thus, $\mu(s)$ satisfies:

\begin{align}
& \mu(s)-\nabla \cdot \frac{r_n^2}{8} \nabla \mu(s)=r_n \sqrt{\frac{\pi}{2}} \sigma_{\mu} \mathcal{W}(s), \quad\text { for } s \in \Omega_n \nonumber \\
& \mu(s)-\nabla \cdot \frac{r_b^2}{8} \nabla \mu(s)=r_b \sqrt{\frac{\pi}{2}} \sigma_{\mu} \mathcal{W}(s), \quad\text { for } s \in \Omega_b, \label{eq2}
\end{align}



where $\Omega_n$ and $\Omega_b$ are normal and barrier regions, respectively, and their union covers the study area $\Omega$. Parameters $r_n$ and $r_b$ denote the spatial range in normal and barrier regions, with $r_b = r_n/h$ (usually with $h$ large, e.g., $h=10$) to reduce correlation across barriers. Here, $\nabla = \left(\frac{\partial}{\partial x}, \frac{\partial}{\partial y}\right)$ and $\mathcal{W}(s)$ denotes Gaussian white noise.

\subsection{Transparent Barrier Model}

The Transparent Barrier model follows the concept of the Barrier model, initially proposed for non-stationary Gaussian fields in the presence of physical barriers. First, a stochastic partial differential equation (SPDE) is formulated to describe a spatial Gaussian field with Matérn correlation. Second, the Matérn field covariance construction relies on a collection of all paths between points rather than solely the Euclidean distance.

The critical difference between the Barrier model and the Transparent Barrier model is that the Barrier model eliminates paths crossing barriers entirely, whereas our proposed model weakens these paths based on barrier permeability.

%\subsection{INLA-SPDE framework}

The Transparent Barrier model is defined for a study region $\Omega$, that is partitioned into multiple barrier regions $\Omega_{b_i}$ with different permeability. 

Thus, the system in \eqref{eq2} can be extended as follows:

\begin{align}
& \mu(s)-\nabla \cdot \frac{r_n^2}{8} \nabla \mu(s)=r_n \sqrt{\frac{\pi}{2}} \sigma_{\mu} \mathcal{W}(s), \quad\text { for } s \in \Omega_n \nonumber \\  
& \mu(s)-\nabla \cdot \frac{r_{b_1}^2}{8} \nabla \mu(s)=r_{b_1} \sqrt{\frac{\pi}{2}} \sigma_{\mu} \mathcal{W}(s), \quad\text { for } s \in \Omega_{b_1} \nonumber \\
& \mu(s)-\nabla \cdot \frac{r_{b_2}^2}{8} \nabla \mu(s)=r_{b_2} \sqrt{\frac{\pi}{2}} \sigma_{\mu} \mathcal{W}(s), \quad\text { for } s \in \Omega_{b_2} \nonumber \\
& \vdots \quad\quad\quad\quad\quad\quad\quad\quad\quad\quad \vdots \nonumber \\ 
& \mu(s)-\nabla \cdot \frac{r_{b_l}^2}{8} \nabla \mu(s)=r_{b_l} \sqrt{\frac{\pi}{2}} \sigma_{\mu} \mathcal{W}(s), \quad\text { for } s \in \Omega_{b_l} \label{eq3}
\end{align}

In the INLA framework, computational efficiency and accuracy rely on sparse precision matrices represented by $\mathbf{Q} = \mathbf{\Sigma}^{-1}$. To obtain a sparse $\mathbf{Q}$ and approximate the Gaussian field $u(s)$, the Transparent Barrier model employs the Finite Element Method (FEM) to solve the underlying SPDE, requiring a Delaunay triangulation mesh that discretizes the spatial domain into triangles, as in the case of the stationary Matern approximation.

At each mesh node, we define a linear finite element basis function $\psi_i(s)$, equal to $1$ at node $i$ and $0$ elsewhere. The approximation of the Gaussian field $\tilde{u}(s)$ is then given by

\begin{equation}
\tilde{u}(s) = \sum_{i=1}^{n}u_i\psi_i(s),
\end{equation}

where $u_i$ are Gaussian-distributed coefficients with precision matrix $\mathbf{Q}$.

Rewriting the SPDE gives

\begin{equation}
\left[1 - \nabla \frac{r(s)^2}{8}\nabla\right]u(s) = r(s)\sqrt{\frac{\pi}{2}}W(s),
\end{equation}

with a spatially varying range parameter $r(s)$, partitioned into subdomains $\Omega_q$, each with constant range $r_q$.

Using finite elements, the SPDE is reformulated in weak form as:

\begin{equation}
\left\langle\psi_j,\left[1 - \nabla\frac{r(\cdot)^2}{8}\nabla\right]\tilde{u}\right\rangle = \left\langle\psi_j, r(\cdot)\sqrt{\frac{\pi}{2}}W(\cdot)\right\rangle.
\end{equation}

Finite element matrices are defined as follows:

\begin{equation}
J_{i,j}=\langle\psi_i,\psi_j\rangle,\quad
(D_q)_{i,j}=\langle 1_{\Omega_q}\nabla\psi_i,\nabla\psi_j\rangle,\quad
(C_q)_{i,i}=\langle 1_{\Omega_q}\psi_i,1\rangle.
\end{equation}

This yields a system of equations $A\tilde{u}=\epsilon$ with

\begin{equation}
A = J - \frac{1}{8}\sum_{q=1}^{k}r_q^2D_q,
\end{equation}

and $\epsilon$ Gaussian-distributed with covariance matrix approximated as

\begin{equation}
\mathrm{Cov}(\epsilon)\approx\tilde{C}=\frac{\pi}{2}\sum_{q=1}^{k}r_q^2\tilde{C}_q.
\end{equation}

Thus, we derive the sparse precision matrix $Q=A\tilde{C}^{-1}A$, essential for efficient computation in large-scale spatial modeling.

For the marginal standard deviation of the field at a given location \( s \), we retain the spatially varying standard deviation as derived from solving the SPDE. This approach naturally results in increased prior uncertainty in regions such as narrow inlets, where spatial constraints influence movement patterns. The non-uniform marginal standard deviation aligns with real-world spatial processes—areas with restricted movement exhibit greater variation in occupancy, as they are either highly concentrated or nearly unoccupied. This interpretation emerges naturally from SPDE framework, where the modeled field represents the expected location of a randomly moving entity constrained by the differential operator. 

\subsection{Discretization of the study area - the mesh}

In non-stationary spatial contexts involving physical barriers, the mesh explicitly represents islands and other barriers as internal boundaries, ensuring accurate modeling of varying barrier permeability. Proper boundary specification is critical, as it directly influences the representation of spatial dependencies and model accuracy. Mesh construction, including the adequate representation of boundaries, remains an essential user-assessed step; however, this assessment is similarly required when applying stationary spatial Gaussian fields (Bakka et al., 2019).

The mesh examples constructed placed the outer boundary far from the area of interest to avoid boundary effects (Bakka et al., 2019). Figure *ref* shows a coarse version of the mesh examples in *fig*. The study area is enclosed by a dashed line and the barriers areas by a red line. Two barrier configurations are used to illustrate the mesh and following simulations. The first configuration represents a study area divided by a barrier that has a canal in the middle connecting the top and bottom of the normal area. The second configuration represents a study area completely divided by a thin barrier.

\begin{figure}[h]
\centering
\includegraphics[width=0.5\textwidth]{overleaf/mesh.png}
\caption{Coarse mesh example with barriers in red. On the left the outer polygon shows the extended mesh, and on the right the region is zoomed in so it only shows the area of interest.
Row one shows the configuration with two barriers and a canal, and row two the configuration with one barrier dividing the study area.}\label{fig1}
\end{figure}


\subsection{Correlation}

The transparent barrier permeability is determined by the barrier range parameter, defined as a fraction of the spatial range in the normal area. Higher range fraction values increase barrier permeability, thereby strengthening spatial connections across barriers. Figure \ref{fig2} illustrates how barriers transition from impermeable to completely permeable as the range fraction increases. Specifically, the first row corresponds to a fully impermeable scenario, which can be modeled using the existing Barrier model, whereas the last row corresponds to the stationary model scenario with no barrier effects. To capture correlation structures for all intermediate scenarios between these two extremes, the Transparent Barrier model becomes necessary.

When encountering a barrier, the distribution of possible locations for a moving point is influenced by the barrier's permeability. Under the Barrier model, the probability of finding the point on the opposite side of the barrier is nearly non-existent (Figure \ref{fig2}, top right) unless there is a path around the barrier, such as a canal or gap, where some correlation may still occur (Figure \ref{fig2}, top left). In contrast, under a stationary Gaussian random field, the spatial correlation structure remains unaffected by any barriers, preserving consistent correlations regardless of the barrier's presence or the point's location within the study area (bottom row of Figure 1).

The Transparent Barrier model aims to distort the spatial correlation that the moving point would exhibit under a stationary scenario, but without reaching the extreme behavior of the fully impermeable scenario. This distortion, controlled by adjusting the fraction of the range parameter used in the barrier area, depends on the specific characteristics and nature of barriers encountered in real-world applications.

\begin{figure}[h]
\centering
\includegraphics[width=1\textwidth]{overleaf/corr2points.png}
\caption{Correlation plots illustrating spatial dependence for two barrier configurations. The first configuration (left side) represents a study area divided by a barrier containing a canal in the middle, connecting the upper and lower sections of the normal area. The second configuration (right side) depicts a study area completely divided by a thin barrier. Rows correspond to different barrier permeability levels, expressed by range fractions: 0.01 (row 1), 0.2 (row 2), 0.3 (row 3), 0.5 (row 4), 0.7 (row 5), 0.8 (row 6), and 1 (row 7). Columns correspond to distinct reference points from which spatial correlation is measured across the study areas.}\label{fig2}
\end{figure}

\section{Simulation study}

The simulation involved defining distinct SPDE models for the normal area and for the barrier regions. A spatial range of either 2 (Figure) or 4 (Figure) was used for the normal area in the simulations, and the range parameter in the barrier region was set to be a fraction of this value. Specifically, the normal range was reduced by 0.01, 0.3, 0.5, 0.7, and 0.8, effectively weakening spatial correlation across barrier polygons. The precision matrix derived from the mesh and SPDE models was used to simulate spatial fields from a Gaussian distribution. Observational noise was then added to emulate real-world scenarios, resulting in simulated spatial data that accurately reflect the underlying non-stationary processes due to physical barriers.

To evaluate model performance, spatial fields were simulated using the true spatial structure and then fitted using three distinct approaches: the **stationary model**, which assumes no barriers; the **original Barrier model**, which imposes complete impermeability across barriers; and the **Transparent Barrier model**, which allows for varying permeability by adjusting the barrier range parameter. This setup enables the assessment of each model’s ability to recover the true underlying spatial field across different levels of barrier permeability.

Simulation results in Figure 4 and Figure 5 correspond to the configurations in which the study area is divided by two barriers with a canal connecting the upper and lower sections of the domain, and a study area divided by two connected barriers with no canal in the middle.

In all simulations, the barrier on the left is kept impermeable, while the barrier on the right has varying permeability, controlled by adjusting the range fraction in the SPDE formulation. Each row in Figure 4 and Figure 5 corresponds to a different range fraction on the right side of the barrier, ranging from 0.01 to 0.8, used to simulate the true spatial field. For each simulation, we show the posterior mean of the spatial field on the left, the posterior standard deviation in the middle, and the posterior distribution of the spatial range of the normal area on the right. Results from the Transparent Barrier model are displayed in larger plots, with corresponding results from the Barrier model and stationary model shown in smaller panels to the right —Barrier model on top and stationary model below— facilitating direct visual comparison across models.

The posterior spatial field shows clear differences across models and permeability levels. At low permeability (e.g., range fraction 0.01), the Transparent Barrier model yields results identical to those of the original Barrier model, with a sharp discontinuity across the barrier. As permeability increases (range fractions 0.3–0.8), the Transparent Barrier model captures more spatial continuity across the barrier, contrasting with the original Barrier model, which maintains a discontinuity regardless of the true range, and the stationary model, which overly smooths across barriers. When the range fraction for the barrier on the right is close to 1, the Transparent Barrier model produces results very similar to those of the classical stationary model. However, these are not exactly the same, as the stationary model does not account for the presence of the impermeable left-side barrier (e.g., range fraction fixed at 0.01).


The posterior standard deviation follows a similar trend. As permeability increases, the Transparent Barrier model distributes uncertainty across the barrier more realistically than the other two models, which either overstate the standard deviation (Barrier) or understate it (stationary). This phenomenon is especially pronounced at the edge of the barrier. Higher standard deviation near the barrier when permeability is low can be understood by analogy with a moving individual who becomes "trapped" upon reaching an impermeable boundary, limiting its potential movement and resulting in longer residence times near the barrier relative to more open regions. Moreover, individuals are far less likely to cross the barrier than to move along the same region (either the upper or lower side of the barrier). When the range fraction is close to 0.01, crossing is effectively impossible. Uncertainty is then explained because the concentration of individuals at the edge of the barrier is either very high or very low.

Regarding the posterior distribution of the spatial range, the stationary model tends to underestimate the range in the normal area because it does not distinguish between regions of differing permeability. By averaging over both the barrier and normal areas—despite the lower correlation induced by the barrier—the model produces a posterior estimate biased toward a lower range. In contrast, the Barrier model overestimates the range, likely compensating for its assumption of complete impermeability, which suppresses correlation across the barrier entirely. These estimation biases are mitigated in the Transparent Barrier model, which allows for spatially varying range parameters and provides posterior estimates more consistent with the true underlying structure. Additional simulation examples supporting this behavior are provided in the supplementary material.

These results confirm the value of the Transparent Barrier model in bridging the gap between fully impermeable and fully stationary assumptions, providing more realistic spatial inference in domains where barriers of varying permeability are present.

\begin{figure}[h]
\centering
\includegraphics[width=0.8\textwidth]{overleaf/geom1u.sd.r.png}
\caption{Simulation results for the configuration with a canal connecting the upper and lower sections of the domain. Each row corresponds to a different range fraction (0.01, 0.3, 0.5, 0.7, and 0.8) used in the simulation for the barrier on the right, while the left barrier remains impermeable. From left to right, the columns show: (1) the posterior mean of the spatial field,  (2) the posterior standard deviation, and
(3) the posterior distribution of the spatial range in the normal area.  For each setting, the Transparent Barrier model is displayed in large plots on the left, followed by smaller plots for the Barrier model (top) and stationary model (bottom)}\label{fig3}
\end{figure}

 

\begin{figure}[h]
\centering
\includegraphics[width=0.8\textwidth]{overleaf/geom1u.sd.r.png}
\caption{Posterior results for the configuration with a canal connecting the upper and lower sections of the domain. Each row corresponds to a different range fraction (0.01, 0.3, 0.5, 0.7, and 0.8) used in the simulation for the barrier on the right, while the left barrier remains impermeable. From left to right, the columns show: (1) the posterior mean of the spatial field, (2) the posterior standard deviation, and (3) the posterior distribution of the spatial range in the normal area. For each setting, the Transparent Barrier model is displayed in large plots on the left, followed by smaller plots for the Barrier model (top) and stationary model (bottom).}\label{fig4}
\end{figure}


\begin{figure}[h]
\centering
\includegraphics[width=1\textwidth]{overleaf/geom2u.sd.r.png}
\caption{Posterior results for the configuration with two connected barriers and no canal. Each row represents a different range fraction (0.01, 0.3, 0.5, 0.7, and 0.8) used to simulate the permeability of the barrier on the right, with the left barrier remaining fully impermeable. Columns represent, from left to right: (1) the posterior mean of the spatial field, (2) the posterior standard deviation, and (3) the posterior distribution of the spatial range in the normal area. As in Figure 4, each model’s results are arranged with the Transparent Barrier model shown in large panels, and the Barrier model and stationary model in smaller panels to the right for comparison.}\label{fig5}
\end{figure}

\section{Application to Dugong species distribution in the Red Sea}


\section{Discussion}


%%=============================================%%
%% For submissions to Nature Portfolio Journals %%
%% please use the heading ``Extended Data''.   %%
%%=============================================%%

%%=============================================================%%
%% Sample for another appendix section			       %%
%%=============================================================%%

%% \section{Example of another appendix section}\label{secA2}%
%% Appendices may be used for helpful, supporting or essential material that would otherwise 
%% clutter, break up or be distracting to the text. Appendices can consist of sections, figures, 
%% tables and equations etc.



%%===========================================================================================%%
%% If you are submitting to one of the Nature Portfolio journals, using the eJP submission   %%
%% system, please include the references within the manuscript file itself. You may do this  %%
%% by copying the reference list from your .bbl file, paste it into the main manuscript .tex %%
%% file, and delete the associated \verb+\bibliography+ commands.                            %%
%%===========================================================================================%%

\bibliography{sn-bibliography}% common bib file
%% if required, the content of .bbl file can be included here once bbl is generated
%%\input sn-article.bbl

\end{document}
